\documentclass{beamer}
\mode<presentation>
\usepackage{amsmath}
\usepackage{amssymb}
%\usepackage{advdate}
\usepackage{adjustbox}
\usepackage{subcaption}
\usepackage{enumitem}
\usepackage{multicol}
\usepackage{listings}
\usepackage{url}
\def\UrlBreaks{\do\/\do-}
\usepackage{listings}
\usepackage{url}
\usepackage{circuitikz}
\def\UrlBreaks{\do\/\do-}
\usetheme{Singapore}
\usecolortheme{dolphin}
\setbeamertemplate{footline}
{
  \leavevmode%
  \hbox{%
  \begin{beamercolorbox}[wd=\paperwidth,ht=2.25ex,dp=1ex,right]{author in head/foot}%
    \insertframenumber{} / \inserttotalframenumber\hspace*{2ex}
  \end{beamercolorbox}}%
  \vskip0pt%
}
\setbeamertemplate{navigation symbols}{}

\providecommand{\nCr}[2]{\,^{#1}C_{#2}} % nCr
\providecommand{\nPr}[2]{\,^{#1}P_{#2}} % nPr
\providecommand{\mbf}{\mathbf}
\providecommand{\pr}[1]{\ensuremath{\Pr\left(#1\right)}}
\providecommand{\qfunc}[1]{\ensuremath{Q\left(#1\right)}}
\providecommand{\sbrak}[1]{\ensuremath{{}\left[#1\right]}}
\providecommand{\lsbrak}[1]{\ensuremath{{}\left[#1\right.}}
\providecommand{\rsbrak}[1]{\ensuremath{{}\left.#1\right]}}
\providecommand{\brak}[1]{\ensuremath{\left(#1\right)}}
\providecommand{\lbrak}[1]{\ensuremath{\left(#1\right.}}
\providecommand{\rbrak}[1]{\ensuremath{\left.#1\right)}}
\providecommand{\cbrak}[1]{\ensuremath{\left\{#1\right\}}}
\providecommand{\lcbrak}[1]{\ensuremath{\left\{#1\right.}}
\providecommand{\rcbrak}[1]{\ensuremath{\left.#1\right\}}}
\theoremstyle{remark}
\newtheorem{rem}{Remark}
\newcommand{\sgn}{\mathop{\mathrm{sgn}}}
\providecommand{\abs}[1]{\left\vert#1\right\vert}
\providecommand{\res}[1]{\Res\displaylimits_{#1}}
\providecommand{\norm}[1]{\lVert#1\rVert}
\providecommand{\mtx}[1]{\mathbf{#1}}
\providecommand{\mean}[1]{E\left[ #1 \right]}
\providecommand{\fourier}{\overset{\mathcal{F}}{ \rightleftharpoons}}
%\providecommand{\hilbert}{\overset{\mathcal{H}}{ \rightleftharpoons}}
\providecommand{\system}{\overset{\mathcal{H}}{ \longleftrightarrow}}
%\newcommand{\solution}[2]{\textbf{Solution:}{#1}}
%\newcommand{\solution}{\noindent \textbf{Solution: }}
\providecommand{\dec}[2]{\ensuremath{\overset{#1}{\underset{#2}{\gtrless}}}}
\newcommand{\myvec}[1]{\ensuremath{\begin{pmatrix}#1\end{pmatrix}}}
\let\vec\mathbf

\lstset{
%language=C,
frame=single,
breaklines=true,
columns=fullflexible
}

\numberwithin{equation}{section}

\title{CONTROL SYSTEMS}
\author{ G.VARSHIT \\ EE19BTECH11020}


\date{\today}
\end{frame}
\begin{document}

\begin{frame}
\titlepage
\end{frame}

\begin{frame}

\tableofcontents
\end{frame}
\begin{frame}{Question}
\section{Question}
\frametitle{Question}
\textbf{$\bullet$Find the transfer function $G(s) = V_{o}(s)/V_{i}(s)$, for each operational amplifier circuit shown in figures given below}\\ \\
\begin{circuitikz}
\ctikzset{bipoles/length=1cm}
\draw
(0, 0) node[op amp] (opamp) {}
(opamp.-) to [short,-o] (-2.5,0.35) node[label={[font=\footnotesize]left:$V_{1}$}] {}   -- (-2.5,1) to [R,l^=$R_2$] (-1,1) to [C,l^=$C_2$,-o] (1,1) to (1,0){}
(-2.5,0.35) to [R,l^=$R_1$] (-2.5,-1) to [C,l^=$C_1$,-o] (-2.5,-1.5) to (-2.5,-2) node[ground]{}
[short,-*] (opamp.out) to [short,-o] (1.5,0) node[label={[font=\footnotesize]above:$V_{o}$}] {} 
(opamp.+) to [short,-o] (-1,-0.35)node[label={[font=\footnotesize]below:$V_{i}$}] {}  

;\end{circuitikz}
\begin{circuitikz}
\ctikzset{bipoles/length=1cm}
\draw
(0, 0) node[op amp] (opamp) {}
(opamp.-) to [short,-o] (-2.5,0.35) node[label={[font=\footnotesize]left:$V_{1}$}] {}   -- (-2.5,1) to [R,l^=$R_3$] (-1,1) to [C,l^=$C_2$,-o] (1,1) to (1,0){}
(-1,1) -- (-1,2) to [R,l^=$R_4$] (1,2) -- (1,1)
(-2.5,0.35) to [R,l^=$R_1$] (-2.5,-1) to [C,l_=$C_1$,-o] (-2.5,-2) to (-2.5,-2.5) node[ground]{}
(-2.5,-1) -- (-1.5,-1) to [R,l^=$R_2$] (-1.5,-2) -- (-2.5,-2)
[short,-*] (opamp.out) to [short,-o] (1.5,0) node[label={[font=\footnotesize]above:$V_{o}$}] {} 
(opamp.+) to [short,-o] (-1,-0.35)node[label={[font=\footnotesize]below:$V_{i}$}] {}  
;\end{circuitikz}\\
\hspace{50}Fig.(a)
\hspace{130}Fig.(b)
\end{frame}
\begin{frame}{Non Inverting Op-Amp}
\section{Theoretical background}
It is “the operational amplifier in which the output is in phase with input signal”.The input signal is applied to "+" terminal of Op–Amp:
\begin{center}
\begin{circuitikz}
\ctikzset{bipoles/length=1cm}
\draw
(0, 0) node[op amp] (opamp) {}
(opamp.-) to[R,l_=$Z_1$,-o] (-2.5, 0.35) node[label={[font=\footnotesize]above:$0$}] {} -- (-3, 0.35) to (-3,-0.5) node[ground]{}
(opamp.-) to[short,*-] ++(0,0.5)  coordinate (leftC) node[label={[font=\footnotesize]above:$V_{x}$}] {}
to[R=$Z_2$] (leftC -| opamp.out) node[label={[font=\footnotesize]above:$V_{o}$}] {}
to[short,-*] (opamp.out) to [short,-o] (1.5,0) node[label={[font=\footnotesize]above:$V_{o}$}] {} to (1.5,-0.5) node[ground]{}
(opamp.+) -- (-1.75,-0.35) -- (-1.75,-0.5) to [V,v<=$V_i$] (-1.75,-1) to (-1.75,-1) node[ground]{}
;\end{circuitikz}
\end{center}
Using the voltage divider rule in above circuit :
\begin{equation}
    V_x = \frac{Z_1}{Z_1 + Z_2} * V_{o}
\end{equation}
As the ideal op-amp's input impedance is infinite, its positive and negative terminal are virtually short, implies 
\begin{equation}
    V_{+} = V_{-}
\end{equation}


\end{frame}
\begin{frame}{TRANSFER FUNCTION of Non Inverting Op-Amp}
From the above circuit we know:
\begin{equation}
   V_{+} = V_{i} \ and \ V_{-} = V_{x} 
\end{equation}
So from Eq.(2.2) and Eq.(2.3), this implies
\begin{equation*}
    V_{i} = V_{x}
\end{equation*}
From Eq.(2.1):
\begin{equation*}
    V_{i} = \frac{Z_1}{Z_1 + Z_2} * V_{o}
\end{equation*}
\begin{equation}
    \boxed{\frac{V_o}{V_i} = \frac{Z_1 + Z_2}{Z_1}}
\end{equation}
We apply this formula of transfer function of Non inverting Op-amp [Eq.(2.4)] in Fig.(a) and Fig.(b) to obtain solution.
\end{frame}
\begin{frame}{Resistor and Capacitor values for Fig.(a)}
\section{Solution}
\subsection{Solution(a)}
Here in Fig.(a), resistor and capacitor values are given as following:\\
$\bullet$ $R_{1}$ = 4*10^5  $ \Omega$ \\
$\bullet$ $C_{1}$ = 4*10^{-6} $ F$ \\
$\bullet$ $R_{2}$ = 1.1*10^5 $ \Omega$ \\
$\bullet$ $C_{2}$ = 4*10^{-6} $ F$ \\ 
\vspace{10}
Hence comparing with general terms :\\
1. Impedance $Z_1$ is given by series combination of resistance $R_1$ and capacitance $C_1$ \\
2. Impedance $Z_2$ is given by series combination of resistance $R_2$ and capacitance $C_2$


\end{frame}
\begin{frame}{Calculating $Z_1$ and $Z_2$ for Fig.(a)}
We know Impedance of Capacitor(C) in Laplace form = $\frac{1}{sC}$.\\
Hence, :
\begin{equation*}
    Z_{1}(s) = R_1 + \frac{1}{sC_1}
\end{equation*}   
\begin{equation}
    Z_{1}(s) = 4*10^5 + \frac{1}{4s*10^{-6}}
\end{equation}
\begin{equation*}
    Z_{2}(s) = R_2 + \frac{1}{sC_2}
\end{equation*}
\begin{equation}
    Z_{2}(s) = 1.1*10^5 + \frac{1}{4s*10^{-6}}
\end{equation}

\end{frame}
\begin{frame}{Solution(a)}
From Eq.(2.4) :
\begin{equation*}
    G(s) = \frac{V_{o}(s)}{V_{i}(s)} = \frac{Z_1 + Z_2}{Z_1}
\end{equation*}
From Eq.(3.1) and Eq.(3.2):
\begin{equation*}
    G(s) = \frac{(4*10^5)+(1.1*10^5)+\frac{1}{4s*10^{-6}}+\frac{1}{4s*10^{-6}}}{4*10^5 + \frac{1}{4s*10^{-6}}}
\end{equation*}
Therefore on further simplification:
\begin{equation*}
    G(s) = \frac{51s + 50}{40s + 25}
\end{equation*}
\begin{equation*}
    \boxed{G(s) = 1.275\left(\frac{s + 0.98}{s + 0.625}\right)}
\end{equation*}
   
\end{frame}

\begin{frame}{Resistor and Capacitor values for Fig.(b)}
\subsection{Solution(b)}
Here in Fig.(a), resistor and capacitor values are given as following:\\
$\bullet$ $R_{1}$ = 4*10^5  $ \Omega$ \\
$\bullet$ $C_{1}$ = 4*10^{-6} $ F$ \\
$\bullet$ $R_{2}$ = 6*10^5 $ \Omega$ \\
$\bullet$ $R_{3}$ = 6*10^5 $ \Omega$ \\
$\bullet$ $C_{2}$ = 4*10^{-6} $ F$ \\ 
$\bullet$ $R_{4}$ = 1.1*10^5 $ \Omega$ \\
\vspace{10}
Hence comparing with general terms :\\
1. Impedance $Z_1$ is given by series combination of resistance $R_1$ and another impedance which is parallel combination of $R_2$ and $C_1$\\
2. Impedance $Z_2$ is given by series combination of resistance $R_3$ and another impedance which is parallel combination of $R_4$ and $C_2$
\end{frame}
\begin{frame}{Calculating $Z_1$ and $Z_2$ for Fig.(b)}
We know Impedance of Capacitor(C) in Laplace form = $\frac{1}{sC}$.\\
Hence, :
\begin{equation*}
    Z_{1}(s) = R_1 + \frac{1}{\frac{1}{R_2} + \frac{1}{\frac{1}{sC_1}}} = R_1 + \frac{R_2}{sR_2C_1 + 1}
\end{equation*}   
\begin{equation}
    Z_{1}(s) = 4*10^5 + \frac{6*10^5}{6s*10^5*4*10^{-6} + 1}
\end{equation}
\begin{equation*}
    Z_{2}(s) = R_3 + \frac{1}{\frac{1}{R_4} + \frac{1}{\frac{1}{sC_2}}} = R_3 + \frac{R_4}{sR_4C_2 + 1}
\end{equation*}
\begin{equation}
    Z_{2}(s) = 6*10^5 + \frac{1.1*10^5}{1.1s*10^5*4*10^{-6} + 1}
\end{equation}

\end{frame}
\begin{frame}{Solution(b)}
From Eq.(2.4) :
\begin{equation*}
    G(s) = \frac{V_{o}(s)}{V_{i}(s)} = \frac{Z_1 + Z_2}{Z_1}
\end{equation*}
From Eq.(3.3) and Eq.(3.4):
\begin{equation*}
    G(s) = \frac{(4*10^5)+(6*10^5)+ \frac{6*10^5}{6s*10^5*4*10^{-6} + 1} + \frac{1.1*10^5}{1.1s*10^5*4*10^{-6} + 1}}{4*10^5 + \frac{6*10^5}{6s*10^5*4*10^{-6} + 1}}
\end{equation*}
Therefore on further simplification:
\begin{equation*}
    G(s) = \frac{10.56s^2 + 33.68s + 17.1}{4.224s^2 + 14s + 10}
\end{equation*}

\begin{equation*}
    \boxed{G(s) = \frac{2640s^2 + 8420s + 4275}{1056s^2 + 3500s + 2500}}
\end{equation*}
 
\end{frame}
\begin{frame}{Bode Plot}
\section{Plot}
\subsection{Plot(a)}
This is the Bode plot for the transfer function of Fig.(a)
    \begin{center}
         \includegraphics[height = 7cm]{Transfer_function_1.jpg}
     \end{center}
\end{frame}
\begin{frame}{Response Plots}
These are the respective plots of impulse and step response for Fig.(a)
    \begin{center}
         \includegraphics[height = 4cm]{Impulse_Response_1.jpg}
         \includegraphics[height = 4cm]{Step_Response_1.jpg}
    \end{center}
    
\end{frame}
\begin{frame}{Bode Plot}

\subsection{Plot(b)}
This is the Bode plot for the transfer function of Fig.(b)
    \begin{center}
         \includegraphics[height = 7cm]{Transfer_function_2.jpg}
     \end{center}
    
\end{frame}
\begin{frame}{Response Plots}
These are the respective plots of impulse and step response for Fig.(b)
 \begin{center}
         \includegraphics[height = 4cm]{Impulse_Response_2.jpg}
         \includegraphics[height = 4cm]{Step_Response_2.jpg}
     \end{center}
     
    
\end{frame}

%\begin{frame}
%\frametitle{Introduction}
%\framesubtitle{Literature}
%%\begin{figure}[t!]
%%    \centering
%%    \begin{subfigure}[t]{0.4\columnwidth}
%%        \centering
%%        \includegraphics[width=\columnwidth]{point_source}
%%        \caption{Single point source}
%%\label{fig3:subfig1}        
%%    \end{subfigure}%
%%    ~
%%    \begin{subfigure}[t]{0.4\columnwidth}
%%        \centering
%%        \includegraphics[width=\columnwidth]{pointNoPowerDist_new}
%%        \caption{SNR profile}
%%\label{fig3:subfig2}
%%    \end{subfigure}
%%  %  \caption{Average SNR for a BPP. $N=16$}
%%    \label{fig3}
%%  \end{figure}
%
%\end{frame}
%  
%
%
%%

\end{document}
